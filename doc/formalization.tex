\documentclass[11pt,a4paper]{article}

\usepackage[T1]{fontenc}
\usepackage[utf8]{inputenc}
\usepackage[english]{babel}
\usepackage{fullpage}
\usepackage{lmodern}
\usepackage{amsmath,amssymb}
\usepackage{mathpartir}
\usepackage{xcolor}
\usepackage{booktabs}
\usepackage{csquotes}
\usepackage{biblatex}
\addbibresource{catala.bib}

\newtheorem{theorem}{Theorem}
\newtheorem{lemma}{Lemma}

%% Syntax
\newcommand{\synvar}[1]{\ensuremath{#1}}
\newcommand{\synkeyword}[1]{\textcolor{red!60!black}{\texttt{#1}}}
\newcommand{\synpunct}[1]{\textcolor{black!40!white}{\texttt{#1}}}
\newcommand{\synname}[1]{\ensuremath{\mathsf{#1}}}

\newcommand{\synbool}{\synkeyword{bool}}
\newcommand{\synnum}{\synkeyword{num}}
\newcommand{\syndate}{\synkeyword{date}}
\newcommand{\synvec}{\synkeyword{vec~}}
\newcommand{\synopt}{\synkeyword{opt~}}
\newcommand{\synrule}{\synkeyword{rule~}}
\newcommand{\syncall}{\synkeyword{call~}}
\newcommand{\synscope}{\synkeyword{scope~}}
\newcommand{\synequal}{\synpunct{~=~}}
\newcommand{\synjust}{~\synpunct{:\raisebox{-0.9pt}{-}}~}
\newcommand{\syntyped}{~\synpunct{:}~}
\newcommand{\syndot}{\synpunct{.}~}
\newcommand{\synunit}{\synpunct{()}}
\newcommand{\synunitt}{\synkeyword{unit}}
\newcommand{\syntrue}{\synkeyword{true}}
\newcommand{\synfalse}{\synkeyword{false}}
\newcommand{\synop}{\synpunct{\odot}}
\newcommand{\synlambda}{\synpunct{$\lambda$}~}
\newcommand{\synand}{\synpunct{\wedge}}
\newcommand{\synor}{\synpunct{\vee}}
\newcommand{\synlparen}{\synpunct{(}}
\newcommand{\synrparen}{\synpunct{)}}
\newcommand{\synlsquare}{\synpunct{[}}
\newcommand{\synrsquare}{\synpunct{]}}
\newcommand{\synlbracket}{\synpunct{\{}}
\newcommand{\synrbracket}{\synpunct{\}}}
\newcommand{\synlangle}{\synpunct{$\langle$}}
\newcommand{\synrangle}{\synpunct{$\rangle$}}
\newcommand{\synmid}{\synpunct{~$|$~}}
\newcommand{\synemptydefault}{\synvar{\varnothing}}
\newcommand{\synerror}{\synvar{\circledast}}
\newcommand{\synmerge}{\synkeyword{~++~}}

\newcommand{\synvardef}{\synkeyword{definition~}}
\newcommand{\synscopecall}{\synkeyword{scope\_call~}}

\newcommand{\synarrow}{~\synpunct{$\rightarrow$}~}
\newcommand{\syntypsum}{\synpunct{$+$}}
\newcommand{\syntypproduct}{\synpunct{$\times$}}
\newcommand{\syndiamond}{\;\synpunct{$\diamond$}\;}
\newcommand{\syncomma}{\synpunct{,}}
\newcommand{\synellipsis}{\synpunct{,$\ldots$,}}

\newcommand{\syndef}{$ ::= $}
\newcommand{\synalt}{\;$|$\;}

\newcommand{\synhole}{\synvar{\cdot}}

%% Typing
\newcommand{\typctx}[1]{\textcolor{orange!90!black}{\ensuremath{#1}}}
\newcommand{\typempty}{\typctx{\varnothing}}
\newcommand{\typcomma}{\typctx{,\;}}
\newcommand{\typvdash}{\typctx{\;\vdash\;}}
\newcommand{\typcolon}{\typctx{\;:\;}}
\newcommand{\typlpar}{\typctx{(}}
\newcommand{\typrpar}{\typctx{)}}

%% Evaluation and execution
\newcommand{\exctx}[1]{\textcolor{blue!80!black}{\ensuremath{#1}}}
\newcommand{\exeemptysubdefaults}{\exctx{\mathsf{empty\_count}}}
\newcommand{\execonflictsubdefaults}{\exctx{\mathsf{conflict\_count}}}
\newcommand{\Omegaarg}{\Omega_{arg}}
\newcommand{\excaller}{\exctx{\complement}}
\newcommand{\excomma}{\exctx{,}\;}
\newcommand{\exvdash}{\;\exctx{\vdash}\;}
\newcommand{\exempty}{\exctx{\varnothing}}
\newcommand{\exemptyv}{\exctx{\varnothing_v}}
\newcommand{\exemptyarg}{\exctx{\varnothing_{arg}}}
\newcommand{\exvarmap}{\exctx{~\mapsto~}}
\newcommand{\exscopemap}{\exctx{~\rightarrowtail~}}
\newcommand{\exArrow}{\exctx{~\Rrightarrow~}}
\newcommand{\exeq}{\exctx{\;=\;}}
\newcommand{\exeval}{\exctx{\;\longrightarrow\;}}
\newcommand{\exevalstar}{\exctx{\;\longrightarrow^*\;}}
\newcommand{\exat}{\exctx{\texttt{\;@\;}}}
\newcommand{\exsemicolon}{\exctx{;~}}
\newcommand{\excomp}{\dashrightarrow}


\title{Formalization of the Catala language}
\date{November 2020}
\author{Nicolas Chataing, Denis Merigoux}

\begin{document}
\maketitle

\tableofcontents

\section{Introduction}

Tax law defines how taxes should be computed, depending on various characteristic
of the fiscal household. Government agencies around the world use computer 
programs to compute the law, which are derived from the local tax law. Translating 
tax law into an unambiguous program is tricky because the law is subject to 
interpretations and ambiguities. The goal of the Catala domain-specific language 
is to provide a way to clearly express the interpretation chosen for the 
computer program, and display it close to the law it is supposed to model.

To complete this goal, our language needs some kind of \emph{locality} property
that enables cutting the computer program in bits that match the way the 
legislative text is structured. This subject has been extensively studied by
Lawsky \cite{lawsky2017, lawsky2018, lawsky2020form}, whose work has greatly 
inspired our approach.

The structure exhibited by Lawsky follows a kind of non-monotonic logic called 
default logic \cite{Reiter1987}. Indeed, unlike traditional programming, when the law defines 
a value for a variable, it does so in a \emph{base case} that applies only if 
no \emph{exceptions} apply. To determine the value of a variable, one needs to 
first consider all the exceptions that could modify the base case. 

It is this precise behavior which we intend to capture when defining the semantics 
of Catala.

\section{Default calculus}

We chose to present the core of Catala as a lambda-calculus augmented by a special 
\enquote{default} expression. This special expression enables to deal with 
the logical structure underlying tax law. This lambda-calculus has only unit and 
boolean values, but this base could be enriched with more complex values and traditional 
lambda-calculus extensions (such as algebraic data types).

\subsection{Syntax}
\label{sec:defaultcalc:syntax}

\begin{center}
\begin{tabular}{lrrll}
  Type&\synvar{\tau}&\syndef&\synbool\synalt\synunitt&boolean and unit types\\
  &&\synalt&\synvar{\tau}\synarrow\synvar{\tau}&function type \\
  &&&&\\
  Expression&\synvar{e}&\syndef&\synvar{x}\synalt\syntrue\synalt\synfalse\synalt\synunit&variable, literal\\
  &&\synalt&\synlambda\synlparen\synvar{x}\syntyped\synvar{\tau}\synrparen\syndot\synvar{e}\synalt\synvar{e}\;\synvar{e}&$\lambda$-calculus\\
  &&\synalt&\synvar{d}&default term\\
  &&&&\\
  Default&\synvar{d}&\syndef&\synlangle\synvar{e}\synjust\synvar{e}\synmid $[\synvar{d}^*]$\synrangle&default term\\
  &&\synalt&\synerror&conflict error term\\
  &&\synalt&\synemptydefault&empty error term\\
\end{tabular}
\end{center}

To the syntax of the lambda calculus, we add a construction coming from 
default logic. Particularly, we focus on a subset of default logic called 
categorical, prioritized default logic \cite{Brewka2000}. In this setting, a default is a logical 
rule of the form $A \synjust B$ where $A$ is the justification of the rule and 
$B$ is the consequence. The rule can only be applied if $A$ is consistent with 
our current knowledge. If multiple rules $A \synjust B_1$ and $A \synjust B_2$
can be applied at the same time, then only one of them is applied through 
an explicit order of the rules.

To translate this form of logic inside our programming language, we set $A$ to 
be an expression that can be evaluated to \syntrue{} or \synfalse{}, and $B$
the value that the default should evaluate to if $A$ is true. If $A$ is false,
then we look up for other rules of lesser priority to apply. This priority 
is encoded trough a syntactic tree data structure\footnote{Thanks to Pierre-Évariste Dagand for this insight.}. 

In the term \synlangle\synvar{e_{\text{just}}}\synjust
\synvar{e_{\text{cons}}}\synmid $[\synvar{d}^*]$\synrangle, \synvar{e_{\text{just}}} 
is the justification $A$, \synvar{e_{\text{cons}}} is the consequence $B$ and 
$[\synvar{d}^*]$ is the list of rules to be considered if \synvar{e_{\text{just}}} 
evaluates to \synfalse{}. 
 
Of course, this evaluation scheme can fail if no more 
rules can be applied, or if two or more rules of the same priority have their 
justification evaluate to \syntrue{}. The error terms \synerror{} and \synemptydefault{}
encode these failure cases.

\subsection{Typing}
\label{sec:defaultcalc:typing}

The typing judgment \fbox{$\typctx{\Gamma}\typvdash\synvar{e}\typcolon\synvar{\tau}$} reads as
\enquote{under context $\typctx{\Gamma}$, expression $\synvar{e}$ has type $\synvar{\tau}$}.
\begin{center}
  \begin{tabular}{lrrll}
    Typing context&\typctx{\Gamma}&\syndef&\typempty&empty context\\
    (unordered map)&&\synalt&\typctx{\Gamma}\typcomma\synvar{x}\typcolon\synvar{\tau}&typed variable\\
  \end{tabular}
\end{center}


We start by the usual rules of simply-typed lambda calculus.
\begin{mathpar}
  \inferrule[UnitLit]{}{ 
      \typctx{\Gamma}\typvdash\synunit\syntyped\synunitt
  }

  \inferrule[TrueLit]{}{ 
      \typctx{\Gamma}\typvdash\syntrue\syntyped\synbool
  }

  \inferrule[FalseLit]{}{ 
      \typctx{\Gamma}\typvdash\synfalse\syntyped\synbool
  }

  \inferrule[Var]{}{
      \typctx{\Gamma}\typcomma\synvar{x}\typcolon\synvar{\tau}\typvdash\synvar{x}\syntyped\synvar{\tau}
  }

  \inferrule[Abs]
  {\typctx{\Gamma}\typcomma\synvar{x}\typcolon\synvar{\tau}\typvdash\synvar{e}\typcolon\synvar{\tau'}}
  {\typctx{\Gamma}\typvdash\synlambda\synlparen\synvar{x}\syntyped{\tau}\synrparen\syndot\synvar{e}\typcolon\synvar{\tau}\synarrow\synvar{\tau'}}

  \inferrule[App]
  {
    \typctx{\Gamma}\typvdash\synvar{e_1}\typcolon\synvar{\tau_2}\synarrow\synvar{\tau_1}\\
    \typctx{\Gamma}\typvdash\synvar{e_2}\typcolon\synvar{\tau_2}
  }
  {\typctx{\Gamma}\typvdash\synvar{e_1}\;\synvar{e_2}\typcolon\synvar{\tau_1}}
\end{mathpar}

Then we move to the special default terms. First, the error terms that stand for 
any type.
\begin{mathpar}
  \inferrule[ConflictError]{}{\typctx{\Gamma}\typvdash\synerror\typcolon\synvar{\tau}}

  \inferrule[EmptyError]{}{\typctx{\Gamma}\typvdash\synemptydefault\typcolon\synvar{\tau}}
\end{mathpar}

Now the interesting part for the default terms. As mentioned earlier, the 
justification \synvar{e_{\text{just}}} is a boolean, while \synvar{e_{\text{cons}}}
can evaluate to any value. \TirName{DefaultBase} specifies how the tree structure 
of the default should be typed.
\begin{mathpar}
  \inferrule[DefaultBase]
  {
    \typctx{\Gamma}\typvdash\synvar{e_{\text{just}}}\typcolon\synbool\\
    \typctx{\Gamma}\typvdash\synvar{e_{\text{cons}}}\typcolon\synvar{\tau}\\
    \forall i\in[\![1;n]\!],\;\typctx{\Gamma}\typvdash\synvar{d_i}\typcolon{\tau}
  }
  {\typctx{\Gamma}\typvdash\synlangle\synvar{e_{\text{just}}}\synjust\synvar{e_{\text{cons}}}\synmid 
  \synvar{d_1}\synellipsis\synvar{d_n}\synrangle\typcolon\synvar{\tau}}
\end{mathpar}

The situation becomes more complex in the presence of functions. Indeed, want 
our default expressions to depend on parameters. By only allowing \synvar{e_{\text{just}}}
to be \synbool{}, we force the user to declare the parameters in a \synlambda 
that wraps the default from the outside. Using this scheme, all the expressions 
inside the tree structure of the default will depend on the same bound variable 
\synvar{x}.

However, this scheme is not adapted to the way Catala will expose defaults 
to the user. Indeed, each legislative article will yield a rule that will be 
encoded into a \synvar{e_{\text{just}}}\synjust\synvar{e_{\text{cons}}} expression.
Later, all these expressions will be collected and turned into the tree data 
structure according to the priorities defined by the law. But if each 
\synvar{e_{\text{just}}}\synjust\synvar{e_{\text{cons}}} expression depends on 
a parameter, the source code will include bindings for free variables in each 
of the default tree node. Rather than to hoist these bindings outside of the 
default and perform $\alpha$-renaming, we prefer to explicitly allow bindings 
inside the default tree nodes. Hence, we include a dedicated typing rule, 
\TirName{DefaultFun}, as well as a later rule for performing $\beta$-reduction 
under the default.
\begin{mathpar}
  \inferrule[DefaultFun]
  {
    \typctx{\Gamma}\typcomma\synvar{x}\typcolon\synvar{\tau}\typvdash\typvdash\
    \synlangle\synvar{e_{\text{just}}}\synjust\synvar{e_{\text{cons}}}\synmid\synrangle
    \typcolon\synvar{\tau'}\\
    \forall i\in[\![1;n]\!],\;
    \typctx{\Gamma}\typvdash\synvar{d_i}\typcolon\synvar{\tau}\synarrow\synvar{\tau'}
  }
  {\typctx{\Gamma}\typvdash
  \synlangle\synlambda\synlparen\synvar{x}\syntyped\synvar{\tau}\synrparen\syndot\synvar{e_{\text{just}}}\synjust
  \synlambda\synlparen\synvar{x}\syntyped\synvar{\tau}\synrparen\syndot\synvar{e_{\text{cons}}}\synmid
  \synvar{d_1}\synellipsis\synvar{d_n}\synrangle\typcolon\tau\synarrow\tau'}
\end{mathpar}

\subsection{Evaluation}

We give this default calculus small-step, structured operational semantics. The 
one-step reduction judgment is of the form \fbox{\synvar{e}\exeval\synvar{e'}}.

In our simple language, values are just booleans or functions and we use a evaluation contexts. 
Evaluation contexts are contexts where the hole can only occur at some
admissible positions that often described by a grammar.
\begin{center}
  \begin{tabular}{lrrll}
    Values&\synvar{v}&\syndef&\synlambda\synlparen\synvar{x}\syntyped\synvar{\tau}\synrparen\syndot\synvar{e}&functions\\
                    &&\synalt&\syntrue\synalt\synfalse & booleans\\
                    &&\synalt&\synerror\synalt\synemptydefault&errors\\
    Evaluation &\synvar{C_\lambda}&\syndef&\synhole\;\synvar{e}\synalt\synvar{v}\;\synhole&function application\\
       contexts&&\synalt&\synlangle\synhole\synjust\synvar{e}\synmid \synvar{d_1}\synellipsis\synvar{d_n}\synrangle&default justification evaluation\\
                    &&\synalt&\synlangle\syntrue\synjust\synhole\synmid \synvar{d_1}\synellipsis\synvar{d_n}\synrangle&default consequence evaluation\\
               &\synvar{C}&\syndef&\synvar{C_\lambda}&regular contexts\\  
                    &&\synalt&\synlangle\synfalse\synjust\synvar{e}\synmid \synvar{v_1}\synellipsis\synvar{v_{i-1}}
                    \syncomma\synhole\syncomma\synvar{d_{i+1}}\synellipsis\synvar{d_n}\synrangle&sub-default evaluation
  \end{tabular}
\end{center}

Note that our language is call-by-value.
First, we present the usual reduction rules for beta-reduction 
(function application and default application) 
and evaluation inside a context hole.
\begin{mathpar}
   \inferrule[Context]
  {\synvar{e}\exeval\synvar{e'}}
  {\synvar{C}[\synvar{e}]\exeval\synvar{C}[\synvar{e'}]}

 \inferrule[$\beta_v$]{}{
   (\synlambda\synlparen\synvar{x}\syntyped\synvar{\tau}\synrparen\syndot{e})\;\synvar{v}
    \exeval\synvar{e}[\synvar{x}\mapsto\synvar{v}]
 }

   \inferrule[$\beta_d$]{
     \synvar{d_1}\;\synvar{v}\exeval \synvar{d_1'}\\
     \cdots\\
     \synvar{d_n}\;\synvar{v}\exeval \synvar{d_n'}
   }{
     \synlangle \synlambda\synlparen\synvar{x}\syntyped\synvar{\tau}\synrparen\syndot\synvar{e_{\text{just}}}\synjust
     \synlambda\synlparen\synvar{x}\syntyped\synvar{\tau}\synrparen\syndot\synvar{e_{\text{cons}}}\synmid 
     \synvar{d_1}\synellipsis\synvar{d_n}\synrangle\;\synvar{v}\exeval\\
      \synlangle \synvar{e_{\text{just}}}[\synvar{x}\mapsto\synvar{v}]\synjust
     \synvar{e_{\text{cons}}}[\synvar{x}\mapsto\synvar{v}]\synmid 
     \synvar{d_1'}\synellipsis\synvar{d_n'}\synrangle
     }
\end{mathpar}

Now we have to describe how the default terms reduce. Thanks to a the 
\TirName{Context} rule, we can suppose that the justification of the default 
is already reduced to a variable \synvar{v}. By applying the beta-reduction 
rule $\beta_d$, we can further reduce to the case where \synvar{v} is a boolean.
This is where we encode our default logic evaluation semantics. If \synvar{v} is 
\syntrue{}, then this rule applies and we reduce to the consequence. We don't 
even have to consider rules of lower priority lower in the tree.
\begin{mathpar}
  \inferrule[DefaultJustifTrue]
  {}
  {\synlangle \syntrue\synjust \synvar{e}\synmid \synvar{d_1}\synellipsis\synvar{d_n}\synrangle\exeval e}
\end{mathpar}

If \synvar{v} is \synfalse{}, then we have to consider rules of lower priority
\synvar{d_1}\synellipsis\synvar{d_n} that should be all evaluated (left to right),
according to the sub-default evaluation context. Then, we consider all the 
values yielded by the sub-default evaluation and define two functions over these 
values. Let $\exeemptysubdefaults(\synvar{v_1}\synellipsis\synvar{v_n})$ returns 
the number of empty error terms \synemptydefault{} among the values. We then case analyze on this count:
\begin{itemize}
  \item if $\exeemptysubdefaults(\synvar{v_1}\synellipsis\synvar{v_n}) =n$, then 
  none of the sub-defaults apply and we return \synemptydefault;
  \item if $\exeemptysubdefaults(\synvar{v_1}\synellipsis\synvar{v_n}) =n - 1$,
  then only only one of the sub-default apply and we return its corresponding value;
  \item if $\exeemptysubdefaults(\synvar{v_1}\synellipsis\synvar{v_n}) < n - 1$,
  then two or more sub-default apply and we raise a conflict error \synerror.
\end{itemize}


\begin{mathpar}
  \inferrule[DefaultJustifFalseNoSub]
  {}
  {\synlangle \synfalse\synjust \synvar{e}\synmid \synemptydefault{}\synellipsis\synemptydefault{}\synrangle\exeval \synemptydefault{}}

  \inferrule[DefaultJustifFalseOneSub]
  {}
  {\synlangle \synfalse\synjust \synvar{e}\synmid 
  \synemptydefault\synellipsis\synemptydefault\syncomma\synvar{v}\syncomma\synemptydefault\synellipsis\synemptydefault\synrangle\exeval \synvar{v}}

  \inferrule[DefaultJustifFalseSubConflict]
  {\exeemptysubdefaults(\synvar{v_1}\synellipsis\synvar{v_n}) <n - 1}
  {\synlangle \synfalse\synjust \synvar{e}\synmid \synvar{v_1}\synellipsis\synvar{v_n}\synrangle\exeval \synerror{}}
\end{mathpar}

Last, we need to define how our error terms propagate. Because the rules for 
sub-default evaluation have to count the number of error terms in the list 
of sub-defaults, we cannot always immediately propagate the error term in 
all the evaluation contexts as it usually done. Rather, we rely on the 
distinction between the $\lambda$-calculus evaluation contexts $\synvar{C_\lambda}$
and the sub-default evaluation context. Hence the following rules for error 
propagation.

\begin{mathpar}
   \inferrule[ContextEmptyError]
  {\synvar{e}\exeval\synemptydefault}
  {\synvar{C_\lambda}[\synvar{e}]\exeval\synemptydefault}

\inferrule[ContextConflictError]
  {\synvar{e}\exeval\synerror}
  {\synvar{C_\lambda}[\synvar{e}]\exeval\synerror}
\end{mathpar}

\section{Scope language}

Our core default calculus provides a value language adapted to the drafting style 
of tax law. Each article of law will provide one or more rules encoded as 
defaults. But how to collect those defaults into a single expression that 
will compute the result that we want? How to reuse existing rules in different 
contexts?

These question point out the lack of an abstraction structure adapted to 
the legislative drafting style. Indeed, our \synlambda functions are not 
convenient to compose together the rules scattered around the legislative text. 
Moreover, the abstractions defined in the legislative text exhibit a behavior
quite different from \synlambda functions.

First, the blurred limits between abstraction units.
In the legislative text, objects and data are referred in a free variable style.
It is up to us to put the necessary bindings for these free variables, but 
it is not trivial to do so. For that, one need to define the perimeter of 
each abstraction unit, a legislative \emph{scope}, which might encompass multiple 
articles.

Second, the confusion between local variables and function parameters. The 
base-case vs. exception structure of the law also extends between legislative 
scopes. For instance, a scope $A$ can define a variable $x$ to have value $a$, but 
another legislative scope $B$ can \emph{call into} $A$ but specifying that 
$x$ should be $b$. In this setting, $B$ defines an exception for $x$, that 
should be dealt with using our default calculus.

Based on these two characteristic, we propose a high-level \emph{scope language},
semantically defined by its encoding in the default calculus.

\subsection{Syntax}

A scope $S$ is a legislative abstraction unit that can encompass multiple 
articles. $S$ is comprised of multiple rules that define a scope variable $a$ 
to a certain expression under a condition that characterize the base case or 
the exception.

$S$ can also call into another scope $S'$, as a function can call 
into another. These calls are scattered in the legislative texts and have 
to be identified by the programmer. Since $S$ can call $S'$ multiple times
with different \enquote{parameters}, we have to distinguish between these
sub-call and give them different names \synlparen\synvar{S'}\synarrow\synvar{s_1}\synrparen,
 \synlparen\synvar{S'}\synarrow\synvar{s_2}\synrparen, etc.


\begin{center}
\begin{tabular}{lrrll}
  Scope name&\synvar{S}&&&\\
  Location&\synvar{\ell}&\syndef&\synvar{a}&scope variable\\
        &&\synalt&\synlparen\synvar{S}\synarrow\synvar{s}\synrparen\synlsquare\synvar{a}\synrsquare&sub-scope variable\\
  Expression&\synvar{e}&\syndef&\synvar{\ell}&location\\
  &&\synalt&$\cdots$&default calculus expressions\\
  &&&&\\
  Rule&\synvar{r}&\syndef&\synrule\synvar{\ell}\synequal\synlangle\synvar{e}\synjust\synvar{e}\synrangle&variable definition\\
  &&\synalt&\syncall\synlparen\synvar{S}\synarrow\synvar{s}\synrparen&sub-scope call\\
  Scope declaration&\synvar{\sigma}&\syndef&\synscope\synvar{S}\syntyped $[\synvar{r}^*]$&\\

\end{tabular}
\end{center}

\printbibliography

\end{document}
