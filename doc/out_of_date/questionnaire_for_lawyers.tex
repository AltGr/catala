\documentclass[11pt, french, a4paper]{article}

\usepackage[T1]{fontenc}
\usepackage[utf8]{inputenc}
\usepackage[french]{babel}
\usepackage{lmodern}
\usepackage{minted}
\usepackage{textcomp}
\usepackage[hidelinks]{hyperref}
\usepackage[dvipsnames]{xcolor}
\usepackage{fullpage}
\usepackage[many]{tcolorbox}

\fvset{
  commandchars=\\\{\},
  numbers=left,
  framesep=3mm,
  frame=leftline,
  firstnumber=last,
  codes={\catcode`\$=3\catcode`\^=7}
}

\title{
  Proposition d'un langage de formalisation\\
  de textes législatifs :\\
  Questionnaire à l'intention de juristes
}
\author{
  Denis \bsc{Merigoux}\\Inria\and
  Liane \bsc{Huttner}\\Université Panthéon-Sorbonne\and
  Nicolas \bsc{Chataing}\\Inria -- École Normale Supérieure\and
}

\begin{document}
\maketitle
\renewcommand{\contentsname}{Sommaire}
\tableofcontents

\section{Introduction}

Impôts, allocations et pensions de retraite partagent un point commun : les règles de calcul des montants de ces transferts sociaux sont définies dans des textes législatifs. Ce sont ces textes en langage naturel, votés par le parlement ou décrétés par le gouvernement, qui servent de référence aux administrations et entreprises qui ont besoin de calculer ces montants de transferts. Cependant, ces calculs sont largement effectués automatiquement : il existe donc des programmes informatiques qui suivent les règles définies par la loi pour calculer le montant des transferts sociaux. Dès lors se pose la question suivante : comment être sûr que ces programmes informatiques, écrits dans des langages de programmation, implémentent fidèlement les règles décrites en langage naturel dans les textes législatifs ? Après avoir résumé l'état de l'art en la matière, nous proposons et décrivons un processus nouveau permettant d'atteindre une garantie de conformité bien supérieure. Nous illustrons ensuite la manière dont ce processus fonctionne sur l'exemple des allocations familiales. En fin de document, un questionnaire vous permet de nous transmettre vos remarques sur notre démarche et sur le processus.


\subsection{Quelle garantie pour les calculs de transferts sociaux ?}

Impôts, allocations et pensions de retraite partagent un point commun : les règles de calcul des montants de ces transferts sociaux sont définies dans des textes législatifs. Ce sont ces textes en langage naturel, votés par le parlement ou décrétés par le gouvernement, qui servent de référence aux administrations et entreprises qui ont besoin de calculer ces montants de transferts. Cependant, ces calculs sont largement effectués automatiquement : il existe donc des programmes informatiques qui suivent les règles définies par la loi pour calculer le montant des transferts sociaux. Dès lors se pose la question suivante : comment être sûr que ces programmes informatiques, écrits dans des langages de programmation, implémentent fidèlement les règles décrites en langage naturel dans les textes législatifs ? \href{https://github.com/betagouv/mon-entreprise/issues/796#issuecomment-608446936}{Des mots mêmes d'un développeur} du simulateur \href{https://mon-entreprise.fr}{\texttt{mon-entreprise.fr}} pour le calcul des cotisation sociales, cette fidélité est loin d'être acquise:

\begin{quote}\itshape
  Il faut garder en tête que les implémentations des règles socio-fiscales qui existent aujourd'hui (logiciels de paie, simulateurs des conseillers en gestion de patrimoine, outils des URSSAF, etc.) sont loin d'avoir un niveau d'assurance optimal, ou pour le dire plus clairement il y a des erreurs partout. Pour prendre un sujet d'actualité, je ne suis pas sûr qu'il existe beaucoup d'implémentations correctes des indemnités de chômage partiel par exemple (alors que ça concerne au moins 4 millions de salariés). Il a un décalage entre les textes de lois et leur application, et ça représente parfois des différences de quelques centaines de millions d'euros en agrégé (retraite des auto-entrepreneurs ou des artistes-auteurs par exemple). Bref, tout ça pour dire que si l'on vise bien le meilleur niveau de fiabilité possible, il ne faut pas idéaliser le système tel qu'il fonctionne aujourd'hui.
\end{quote}

Il existe plusieurs manières d'obtenir des garanties de fidélité des programmes par rapport aux textes. La manière la plus simple, actuellement utilisée par la plupart des administrations et entreprises, consiste à faire rédiger par des juristes des études de cas où le calcul est détaillé pour une situation individuelle. Les programmeurs informatiques peuvent ainsi vérifier que le détail et le résultat du calcul informatique sur cette situation correspondent bien à ce qu'ont écrit les juristes. La garantie apportée par ces études de cas est proportionnelle à leur nombre et à leur diversité : si une règle de calcul n'est pas couverte par une de ces études de cas, elle ne sera pas couverte pas la garantie.

De manière plus subtile, il ne suffit pas de couvrir toutes les règles de calcul par des études de cas pour obtenir une garantie complète de fidélité. En effet, le comportement de la règle de calcul dépend des valeurs des montants en présence. Pour obtenir une garantie complète de complexité, il faudrait ainsi couvrir par des études de cas tous les comportements possibles de toutes les règles de calcul en présence. Le nombre d'études nécessaire pour atteindre cet objectif, difficile à quantifier, est très élevé et en pratique jamais atteint. Dès lors,comment faire pour obtenir la garantie complète de fidélité du programme informatique par rapport à la loi ?

Plutôt que d'évaluer le programme informatique par l'extérieur avec des études de cas, nous proposons une approche alternative basée sur un vieux concept d'informatique, la programmation littéraire. Le principe de la programmation littéraire est de mêler dans une même source le code informatique ainsi que la description en langage naturel de ce que le code est censé faire. Dans notre cas, cette description en langage naturel est également la source de vérité sur le comportement du programme informatique : les textes législatifs.

Concrètement, cela consiste à partir du texte législatif et annoter ligne à ligne le texte juridique par une traduction informatique fidèle qui retranscrirait tout le contenu sémantique de la ligne de texte, et rien que ce contenu. En procédant ainsi, le problème de la garantie globale de fidélité du programme informatique par rapport à la loi est réduit à la vérification locale de la fidélité d'un morceau de code par rapport à une ligne du texte législatif. Cette vérification demande une grande expertise juridique, car il s'agit de connaître précisément l'interprétation du texte législatif dans toutes les situations possibles.

Notre objectif est de rendre possible cette vérification par des juristes spécialisés. Il faut donc que les annotations sous forme de code soient compréhensibles par ces juristes moyennant une formation minimale. Pour cela, il est nécessaire de créer un nouveau langage de programmation qui répond à cette contrainte de lisibilité. L'annotation des textes législatifs par des programmes rédigés dans ce nouveau langage de programmation correspond au concept de formalisation : il s'agit de donner une définition précise au sens mathématique à toutes les règles de calcul décrites dans la législation.

\subsection{Pour une législation socio-fiscale formalisée}

Nous affirmons que la méthode basée sur la programmation littéraire des textes législatifs que nous proposons apporterait un progrès significatif par rapport aux méthodes actuelles de production des implémentations informatiques de ces textes, c'est à dire du code qui calcule ce que décrit la loi. Ce progrès se décline en trois principaux axes.

\paragraph{Sécurité juridique} Contrairement au langage naturel, un langage de programmation ne laisse pas de place à l'imprécision et à l'incohérence. Lors de l'implémentation d'un programme censé suivre des textes de loi, le programmeur est obligé de faire des décisions arbitraires. Par exemple, si la loi spécifie une règle différente selon que le montant des revenus soit en dessous ou au-dessus d'un certain seuil, le programmeur doit décider arbitrairement ce qui se passe quand le montant des revenus est égal au seuil. La revue du code par un juriste spécialisé permettrait de vérifier que ces choix arbitraires sont des interprétations valides du texte législatif.

Plus globalement, la garantie d'une implémentation en code informatique fidèle au texte législatif apporte un très haut niveau de sécurité juridique à l'organisation qui utiliserait cette technique. Le champ des méthodes formelles, sur lequel repose la théorie derrière cette proposition, est utilisé dans tous les secteurs d'activité critiques tels que le transport (aérien et ferroviaire), la cryptographie, le domaine spatial et même le contrôle des centrales nucléaires. Un contentieux soulevé par une mauvaise implémentation d'un texte législatif peut avoir de lourdes conséquences si cette implémentation est utilisée à grande échelle. Dans ce cas, les processus utilisés en méthodes formelles sont tout à fait adaptés pour prévenir toute possibilité de contentieux vis-à-vis du logiciel informatique.

\paragraph{Cohérence} Ces problèmes de cohérence se posent aussi en droit fiscal et en droit social. Dans ce cas particulier, la cohérence législative repose sur la cohérence mathématique des règles de calcul définies dans les textes. Il s'agit par exemple de vérifier que le montant d'une allocation est bien dégressif par rapport aux revenus du ménage, ou bien que le taux marginal d'imposition ne peut dépasser un certain seuil. La formalisation des règles de calcul permet d'étudier ces règles en tant qu'objets mathématiques, et ainsi de prouver comme théorème la cohérence législative.

Un deuxième aspect de la cohérence législative concerne l'existence même de plusieurs versions du code informatique traduisant un même texte législatif. Si plusieurs organisations possèdent chacune leur programme informatique censé respecter le même texte législatif, comment s'assurer que ces implémentations sont bien cohérentes ? D'un point de vue économique, l'existence même de plusieurs implémentations concurrentes d'un même texte législatif est contre-productive en termes de maintenance. Pour cette raison, l'implémentation à base de programmation littéraire que Il faut garder en tête que les implémentations des règles socio-fiscales qui existent aujourd'hui (logiciels de paie, simulateurs des conseillers en gestion de patrimoine, outils des URSSAF, etc.) sont loin d'avoir un niveau d'assurance optimal, ou pour le dire plus clairement il y a des erreurs partout. Pour prendre un sujet d'actualité, je ne suis pas sûr qu'il existe beaucoup d'implémentations correctes des indemnités de chômage partiel par exemple (alors que ça concerne au moins 4 millions de salariés). Il a un décalage entre les textes de lois et leur application, et ça représente parfois des différences de quelques centaines de millions d'euros en agrégé (retraite des auto-entrepreneurs ou des artistes-auteurs par exemple). Bref, tout ça pour dire que si l'on vise bien le meilleur niveau de fiabilité possible, il ne faut pas idéaliser le système tel qu'il fonctionne aujourd'hui.nous proposons servirait d'implémentation unique de référence du texte législatif, partagée par toutes les organisations qui en auraient besoin. De cette manière, la cohérence de l'interprétation de la loi par l'implémentation informatique serait garantie.

\paragraph{Transparence} La nécessité d'une implémentation unique de référence pour les textes législatifs nous amène naturellement à l'enjeu de la transparence. En effet, pour qu'elle puisse bénéficier à l'ensemble des organisations qui en auraient besoin, l'implémentation de référence doit être publiée en open-source. L'ouverture du code source de ces implémentations considérées comme documents administratifs est une obligation légale depuis la loi 2016-1321 du 7 octobre 2016 pour une République numérique, confirmée par la décision du tribunal administratif de Paris du 18 février 2016 en ce qui concerne l'implémentation du calcul de l'impôt sur le revenu.

L'ouverture du code source possède également l'avantage de permettre un processus collaboratif d'écriture et de validation de l'implémentation entre les diverses organisations concernées. Concrètement, si l'implémentation est publiée par une administration publique, elle pourrait recueillir avant mise en production d'éventuels commentaires ou questions d'organisations privées ou associatives afin de préciser et d'affiner certains choix d'interprétation \emph{ex ante}, évitant ainsi de recourir au contentieux \emph{ex post}.

Pour ces raisons, l'utilisation de notre méthode à base de programmation littéraire nous apparaît comme plus avantageuse à tout point de vue que les méthodes actuelles de validation par études de cas. Les programmes écrits dans le nouveau langage de programmation seront ensuite traduits (« compilés ») vers d'autres langages de programmation plus traditionnels pour être exécutés au sein d'applications informatiques.


\section{Implémentation des allocations familiales}

Afin de mettre en pratique ces nouveaux concepts, nous vous proposons un exemple d'annotation de plusieurs article de loi  définissant les allocations familiales par des morceaux de programmes écrits dans notre nouveau langage de programmation \emph{ad hoc}. Le guide de lecture ci-dessous contient normalement toutes les informations nécessaires à la compréhension des annotations informatiques.

\paragraph{Avertissement} L'étude de cas présentée ci-dessous n'a pas pour but d'être exhaustive. En effet, il manquerait beaucoup d'autres articles de loi pour définir complètement les allocations familiales. De même, certaines partie des articles présentées ne sont pas formalisés pour des raisons précisées dans le texte. La suite a donc une valeur essentiellement illustrative par rapport aux diverses fonctionnalités du langage de programmation et de la manière dont on pourrait vérifier localement la conformité du code par rapport au texte de loi.

\subsection{Guide de lecture}

\providecommand{\kw}[1]{\textbf{\textcolor{OliveGreen}{#1}}}
\providecommand{\inlinekw}[1]{\kw{\texttt{#1}}}
\providecommand{\cm}[1]{\textit{\texttt{\textcolor{PineGreen}{#1}}}}

\paragraph{Objectif} Le but est d’annoter le texte de loi avec un langage informatique correspondant (qu’on appellera « code »). Le langage informatique doit traduire exactement le contenu des articles, sans rien ajouter. Pour une plus grande clarté, le code sera placé dans le texte de loi, aux endroits pertinents. Cela permettra de vérifier localement, ligne à ligne, si le code correspond bien au texte de loi. Ce code sera mis en valeur par une \inlinekw{police différente et une coloration particulière}.

Ce guide a pour objet de définir les termes utilisés dans le code. Chaque terme défini a été choisi pour permettre une compréhension égale des juristes et informaticiens. Par conséquent, chaque terme a une signification précise en droit et en informatique. La signification retenue ici n’est pas exactement celle du juriste ou de l’informaticien, mais renvoie à des notions que l’un et l’autre comprennent. C’est pourquoi cette étape de définition -- ou de syntaxe -- doit être lue attentivement.

\paragraph{Métadonnées} Les métadonnées sont la description de tous les faits présents dans la réalité que va mentionner le morceau de texte de loi. Cette partie du code est nécessaire d'un point de vue informatique afin de donner de la \inlinekw{structure} aux informations manipulées, mais elle est accessoire lorsque l'on veut juste vérifier la cohérence avec le texte de loi.

Les \inlinekw{données} sont les faits présents dans la réalité, dans un contexte donné. Ce sont tous les éléments que le programme va manipuler. Par exemple, dans la proposition suivante « tout enfant jusqu’à la fin de l’obligation scolaire », les données sont (1) l'enfant  et (2) la fin de l’obligation scolaire. Une donnée peut également dépendre d'un paramètre, par exemple un âge limite que dépend \inlinekw{de} l'enfant.

Une donnée est annotée par un \inlinekw{contenu}, qui décrit ce que représente cette donnée : un entier pour un nombre d’années, un montant pour un revenu, etc. Cette annotation de contenu est nécessaire en informatique, car c’est ce qui permet d’effectuer les calculs numériques. Les \inlinekw{conditions} sont un type particulier de données sans contenu, qui correspondent à des concepts juridiques.

% Parfois, une donnée peut avoir un contenu qui prend plusieurs formes. Par exemple, l’entité en charge d’un enfant est soit un couple soit une personne seule. On exprime cette disjonction de cas en utilisant \inlinekw{choix}, puis en introduisant chacun des cas avec \inlinekw{-}\inlinekw{-}. Lors que l’on veut ensuite utiliser une donnée dont le contenu est un choix, il faut alors examiner chacun des cas avec \inlinekw{selon} $\ldots$ \inlinekw{sous forme}.

Autre cas particulier du contenu d’une donnée : si celle-ci fait référence à plusieurs choses. Par exemple, la donnée \texttt{enfants} fait référence à tous les enfants d’un ménage. On exprimera cette multiplicité à l’aide de \inlinekw{collection} et si l’on veut utiliser cette collection de données, on le fera avec \inlinekw{existe} $\ldots$ \inlinekw{dans} $\ldots$ \inlinekw{tel que} et \inlinekw{pour tout} $\ldots$ \inlinekw{dans} $\ldots$ \inlinekw{on a}.

\paragraph{Champ d’application} Le champ d’application d’un article de loi s’inscrit dans un contexte particulier. Dans le code, il est nécessaire de préciser le champ d’application de chaque article de loi. Ce champ d’application sera introduit par \inlinekw{champ d’application}. Il est également nécéssaire de préciser quelles données seront utilisées dans un champ d'application ; cela est fait à la fin de la section métadonnées.

Le texte de loi sera donc saucissonné par des blocs de code, tous introduits par leur champ d'application. À l'intérieur du code, des messages n'ayant pas valeur informatique mais laissés à titre de commentaires seront introduits par \cm{\#} et \cm{écrits dans cette police}.

\paragraph{Règles} Le but du programme est de traduire en langage informatique les règles exprimées dans le texte de loi. Les règles sont donc les éléments du texte législatif qui permettent d’aboutir à des conséquences. Toute règle ne s’applique que si une condition précise est valable. Cette condition est introduite par \inlinekw{sous condition}, et sa conséquence par \inlinekw{conséquence}.

Basiquement, une \inlinekw{règle} permet d’exprimer des relations logiques entre concepts du texte de loi. Elle permet notamment de préciser quand des \inlinekw{conditions} sont remplies. La \inlinekw{définition} permet de définir une donnée présente dans le champ d'application en fonction d’une autre donnée, par exemple pour exprimer une règle de calcul.

Les \inlinekw{assertions} sont, en langage informatique, une condition de cohérence. Le code doit vérifier que les assertions sont toujours vraies lorsqu’il s’exécute. Cela permet d’assurer une cohérence d’ensemble à tout le programme. Les assertions peuvent être très variées et imposer plusieurs types de vérifications.


\subsection{Corpus législatif annoté}

Les numéros de ligne à gauche servent uniquement de référence, leur numérotation
n'est pas consécutive ici car elle respecte celle du fichier source à partir
duquel a été généré ce document.

% \documentclass[11pt, french]{article}

\usepackage[T1]{fontenc}
\usepackage[utf8]{inputenc}
\usepackage{lmodern}
\usepackage{hyperref, csquotes, textcomp}
\usepackage{listings}
\usepackage{fullpage}
\usepackage[french]{babel}
\usepackage[dvipsnames]{xcolor}

\usepackage{draftwatermark}
\SetWatermarkText{Brouillon}
\SetWatermarkScale{1}
\SetWatermarkColor[gray]{0.92}

\makeatletter
\lstdefinelanguage{lawspec}{%
  basicstyle=\ttfamily,
  keywordstyle = [1]{\color{BlueViolet}},
  keywordstyle = [2]{\color{RedViolet}},
  morekeywords = [1]{situation donnee, type, de, donnee, situation,
    collection, regle, condition, consequence, existe, pour,
    tout, dans, tel, que, optionnelle, defini, comme, source, fixe, par,
    assertion, constante, avec, varie, maniere, decroissante, croissante,
    choix, parametre, fonction, renvoie, parametres, on, a},
  morekeywords = [2]{ou, et, si, alors, entier, booléen, montant, cardinal,
   maintenant,  an, mois, selon, sous, forme},
  otherkeywords = {<, >, =, +, *, --,  /, ;, :, .},
  morekeywords = [1]{--, ;, :, .},
  morekeywords = [2]{=, <, >, +, *, /, -},
  % morestring=[b]",
  % sensitive=true,%
  numbers=left,
  firstnumber=last,
  numberstyle=\footnotesize\color{gray},
  % numbersep=4pt,
%  numbers=left,
%   columns=[l]fullflexible,
  texcl=true,
%   mathescape=true,
%  xleftmargin=10pt,
  % identifierstyle={\ttfamily},
% Here is the range marker stuff
  % rangeprefix=(*---\ ,
  % includerangemarker=false,
  % stringstyle=\rmfamily,
  % lineskip=-4pt,
  showspaces=false,
  showstringspaces=false,
  morecomment=[f][\color{gray}][0]{\#},
  commentstyle=\color{gray}\itshape,
  breaklines=false}
\lstset{language=lawspec}
\makeatother

\title{
  Proposition d'un langage de formalisation de textes législatifs :\\
  document à l'intention des juristes
}
\author{
  Denis Merigoux\\Inria\and
  Nicolas Chataing\\Inria -- École Normale Supérieure\and
  Liane Huttner\\Université Panthéon-Sorbonne
}


\begin{document}

\maketitle

\section{Guide de lecture}

\paragraph{Programmation littérale} Le texte de la loi est annoté avec des morceaux de code qui traduisent le contenu des articles et alinéa en termes formels. Ces blocs de code sont intercalés dans le texte de loi, remarquables à leur police en chasse fixe et la coloration de certains mots. Un morceau de code annotant une ligne d'un texte de loi doit contenir tout le contenu sémantique de la ligne de texte, et rien d'autre que ce contenu. L'idée est de pouvoir vérifier localement, ligne à ligne, la cohérence entre le code et le texte législatif.

\paragraph{Situations} Une situation correspond à un contexte logique dans un morceau de code. De même que dans un texte législatif, on se met dans une situation particulière dans laquelle les entités et concepts que l'on nomme ont un sens, chaque morceau de code se place dans un contexte bien particulier auquel on donne un nom. Deux morceaux de code partagent le même contexte logique si ils sont introduits par des lignes \lstinline|situation| ayant le même nom. Une ligne \lstinline|situation| précise aussi la source juridique du contenu qui y sera précisé : loi, décret, règlement ou implicite pour un contenu logique non explicité dans la loi mais qui découle du bon sens.

\paragraph{Données} Au sein d'une situation, il est possible de déclarer différentes données (\lstinline|donnee|). Ces données sont les quantités et prédicats que le programme va manipuler. Afin de préciser ces données, chacune d'entre elle est assortie d'un type ou d'une situation qui en décrit le contenu. Un type correspond à des quantités primitives comme les entiers, les booléens (vrai ou faux), les dates, les montants d'argent, etc. Une donnée peut aussi contenir une situation qui elle même contient d'autre données, permettant de donner de la structure. Lorsque rien n'est précisé quant au contenu d'une donnée, c'est qu'il n'y a pas de contenu (ou plutôt le contenu est trivial).

\paragraph{Règles} Le but de l'algorithme est d'exprimer comment calculer une donnée-but en fonction d'un ensemble de données-entrées. Cependant, les textes ne définissent pas de données-but ni de données-entrées ; ils se contente d'exprimer des relations entre les données. Une des relations exprimées par les textes à propos des données est la définition. Les règles (\lstinline|regle|) permettent de définir une donnée en fonction d'autres données de la situation. À l'instar des textes, cette définition peut être conditionnelle, pour cela on utilisera les mots-clés \lstinline|condition| et \lstinline|consequence|.

\paragraph{Assertions} Le deuxième type de relation que les textes expriment sur les données consiste à exiger une condition de cohérence. Les \lstinline|assertion| permettent d'exprimer cette exigence de cohérence ; ces conditions doivent être toujours vraies lorsque l'algorithme s'exécute. Une assertion peut tout aussi bien imposer une condition sur les données en entrée, qu'imposer une propriété que l'algorithme doit vérifier. Par exemple, il est possible d'exiger dans une assertion que telle donnée varie de manière décroissante en fonction d'une autre donnée.

\paragraph{Collections} Parfois, le contenu d'une donnée correspond à une collection de contenus, au nombre indéterminé. Il est possible de déclarer ces \lstinline|collections|, et l'on peut alors accéder aux éléments individuels des collections en utilisant les formule existentielles (\lstinline|existe ...  tel que|) ou universelles (\lstinline|pour tout ...   on a|).

\paragraph{Choix} Si les \lstinline|situation| permettent de regrouper plusieurs données dans un même contexte, les \lstinline|choix| permettent au contraire d'exprimer une disjonction de cas dans un algorithme. Chaque possibilité du choix peut, à l'instar des données d'une situation, porter un contenu qui peut être un type, une situation ou bien un autre choix. Afin d'examiner dans le code la valeur contenu dans un contenu choix, on utilisera la syntaxe \lstinline|selon ...   sous forme|.



\section{Corpus législatif et réglementaire définissant les allocations familiales}

\paragraph{Article L511-1} Les prestations familiales comprennent :\\
1°) la prestation d'accueil du jeune enfant ;\\
2°) les allocations familiales ;\\
3°) le complément familial ;\\
4°) L'allocation de logement régie par les dispositions du livre VIII du code de la construction et de l'habitation ;\\
5°) l'allocation d'éducation de l'enfant handicapé ;\\
6°) l'allocation de soutien familial ;\\
7°) l'allocation de rentrée scolaire ;\\
8°) (Abrogé) ;\\
9°) l'allocation journalière de présence parentale.
\begin{lstlisting}
choix prestation:
  -- PrestationAccueilJeuneEnfant
  -- AllocationsFamiliales
  -- ComplementFamilial
  -- AllocationLogement
  -- AllocationEducationEnfantHandicape
  -- AllocationSoutienFamilial
  -- AllocationRentreeScolaire
  -- AllocationJournalierePresenceParentale.

situation ContextePrestationsFamiliales :
  donnee prestation_courante de choix prestation.
\end{lstlisting}

\paragraph{Article L512-3} Sous réserve des règles particulières à chaque prestation, ouvre droit aux prestations familiales :
\begin{lstlisting}
situation ContextePrestationsFamiliales source loi :
  donnee droits_ouverts.
\end{lstlisting}
1°) tout enfant jusqu'à la fin de l'obligation scolaire ;
\begin{lstlisting}
situation EnfantPrestationsFamiliales source loi :
  donnee fin_obligation_scolaire de type entier.

situation ContextePrestationsFamiliales source loi :
  donnee enfants collection de situation EnfantPrestationsFamiliales ;
  regle condition
    existe enfant dans enfants tel que
      maintenant < enfant.fin_obligation_scolaire
  consequence droits_ouverts defini.
\end{lstlisting}
2°) après la fin de l'obligation scolaire, et jusqu'à un âge limite, tout enfant dont la rémunération éventuelle n'excède pas un plafond.
\begin{lstlisting}
situation ContextePrestationsFamiliales source loi :
  donnee age_limite_L512_3_2 de type entier ;
  donnee plafond_remuneration_L512_3_2 de type montant.

situation EnfantPrestationsFamiliales source loi :
  donnee age de type entier ;
  donnee remuneration de type montant ;
  donnee qualifie_pour_prestation_sauf_age ;
  regle condition
    maintenant > fin_obligation_scolaire et
    remuneration < plafond_remuneration_L512_3_2
  consequence qualifie_pour_prestation_sauf_age defini ;
  donnee enfant_qualifie_pour_prestation ;
  regle condition
    qualifie_pour_prestation_sauf_age et
    age < age_limite_L512_3_2
  consequence qualifie_pour_prestation defini.

situation ContextePrestationsFamiliales source loi :
  regle condition
    existe enfant dans enfants tel que
      enfant.qualifie_pour_prestation
  consequence droits_ouverts defini.
\end{lstlisting}
Toutefois, pour l'attribution du complément familial et de l'allocation de logement mentionnés aux 3° et 4° de l'article L. 511-1, l'âge limite peut être différent de celui mentionné au 2° du présent article.
\begin{lstlisting}
situation ContextePrestationsFamiliales source loi :
  donnee age_limite_L512_3_2_alternatif de type entier ;
  regle optionnelle condition
    prestation_courante = ComplementFamilial ou
    prestation_courante = AllocationLogement
  consequence
    age_limite_L512_3_2 defini comme age_limite_L512_3_2_alternatif.
\end{lstlisting}

\paragraph{Article L521-1} Les allocations familiales sont dues à partir du deuxième enfant à charge.
\begin{lstlisting}
situation AllocationsFamiliales source loi:
  donnee contexte de situation ContextePrestationsFamiliales ;
  regle contexte.prestation_courante
     defini comme AllocationsFamiliales ;
  donnee allocations_familiales_dues ;
  donnee nombre_enfants_a_charge : entier.
  regle nombre_enfants_a_charge defini comme cardinal(contexte.enfants) ;
  regle condition
    nombre_enfants_a_charge >= 2
  consequence allocations_familiales_dues defini.
\end{lstlisting}
Une allocation forfaitaire par enfant d'un montant fixé par décret est versée pendant un an à la personne ou au ménage qui assume la charge d'un nombre minimum d'enfants également fixé par décret lorsque l'un ou plusieurs des enfants qui ouvraient droit aux allocations familiales atteignent l'âge limite mentionné au 2° de l'article L. 512-3. Cette allocation est versée à la condition que le ou les enfants répondent aux conditions autres que celles de l'âge pour l'ouverture du droit aux allocations familiales.
\begin{lstlisting}
situation AllocationFamiliales source loi :
  donnee allocation_forfaitaire_L521_1 de type montant ;
  assertion allocation_forfaitaire_L521_1 fixe par decret ;
  donnee nombre_minimum_enfants_L521_1 de type entier ;
  assertion nombre_minimum_enfants_L521_1 fixe par decret.

choix entite_en_charge :
  -- FamilleMonoparentale de situation Personne
  -- Couple de situation Menage.

situation AllocationFamiliales source loi :
  donnee entite_en_charge_des_enfants de choix entite_en_charge ;
  constante duree_allocation_familiale de type duree defini comme 1 an.
  donnee propriete allocation_forfaitaire_L521_1_versee ;
  regle conditon
    nombre_enfants_a_charge > nombre_minimum_enfants_L521_1 et
    (existe enfant dans contexte.enfants tel que
      enfant.age = age_limite_L512_3_2 et
      enfant.qualifie_pour_prestation_sauf_age)
  consequence allocation_forfaitaire_L521_1_versee defini.
\end{lstlisting}
Le montant des allocations mentionnées aux deux premiers alinéas du présent article, ainsi que celui des majorations mentionnées à l'article L. 521-3 varient en fonction des ressources du ménage ou de la personne qui a la charge des enfants, selon un barème défini par décret.
\begin{lstlisting}
situation Personne source implicite :
  donnee ressources de type montant.

situation Menage source implicite :
  donnee ressources de type montant ;
  donnee parent1 de situation Personne ;
  donnee parent2 de situation Personne ;
  regle ressources defini comme
    parent1.ressources + parent2.ressources.

situation AllocationFamiliales source loi :
  donnee ressources_entite_en_charge de type montant.
  regle ressources_entite_en_charge defini comme
    selon entite_en_charge_enfants sous forme
    -- Monoparentale de parent : parent.ressources
    -- Couple de menage : menage.ressources ;
  donnee montant_allocations_familiales de type montant ;
  assertion
    montant_allocations_familiales fixe par decret et
    montant_allocations_familiales varie avec ressources_entite_en_charge ;
  assertion
    allocation_forfaitaire_L521_1 fixe par decret et
    allocation_forfaitaire_L521_1 varie avec ressources_entite_en_charge ;
  donnee majorations_512_3 de type montant ;
  assertion
    majorations_512_3 fixe par decret et
    majorations_512_3 varie avec ressources_entite_en_charge.
\end{lstlisting}
Le montant des allocations familiales varie en fonction du nombre d'enfants a charge.
\begin{lstlisting}
situation AllocationFamiliales source loi :
  assertion
    montant_allocations_familiales varie avec de nombre_enfants_a_charge.
\end{lstlisting}
Les niveaux des plafonds de ressources, qui varient en fonction du nombre d'enfants à charge, sont révisés conformément à l'évolution annuelle de l'indice des prix à la consommation, hors tabac.
\begin{lstlisting}
situation AllocationFamiliales source loi :
 donnee plafonds_ressources_allocations_familiales
    collection de type montant ;
 assertion
  pour tout plafond dans plafonds_ressources_allocations_familiales on a
    plafond varie avec nombre_enfants_a_charge.
# TODO: comment parler de l'évolution?
\end{lstlisting}
Un complément dégressif est versé lorsque les ressources du bénéficiaire dépassent l'un des plafonds, dans la limite de montants définis par décret. Les modalités de calcul de ces montants et celles du complément dégressif sont définies par décret.
\begin{lstlisting}
situation AllocationFamiliales source loi :
  donnee complement_degressif_allocations_familiales de type montant ;
  assertion complement_degressif_allocations_familiales varie avec
      nombre_enfants_a_charge de maniere decroissante ;
  assertion
     complement_degressif_allocations_familiales fixe par decret.
\end{lstlisting}

\paragraph{Article L521-2} Les allocations sont versées à la personne qui assume, dans quelques conditions que ce soit, la charge effective et permanente de l'enfant.
\begin{lstlisting}
situation RecipendaireDivise source implicite :
  donnee recipiendaire1 de situation Personne ;
  donnee recipiendaire2 de situation Personne ;

situation EnfantAllocationsFamiliales source loi :
  donnee contexte de situation EnfantPrestationsFamiliales ;
  donnee entite_en_charge_de_l_enfant de choix entite_en_charge.

situation AllocationFamiliales source loi :
  donnee enfants collection de situation EnfantAllocationsFamiliales ;
  regle cardinal(enfants) defini comme cardinal(contexte.enfants) ;
  regle pour tout enfant_contexte, enfants
    dans enfants.contexte, enfants on a
    enfant.contexte defini comme enfant_contexte ;
  regle pour tout enfant dans enfants on a
    enfant.entite_en_charge_de_l_enfant defini comme
      entite_en_charge_des_enfants

choix recipiendaire:
  -- Complet de choix entite_en_charge
  -- Divise de situation RecipendaireDivise.

situation EnfantAllocationsFamiliales source loi :
  donnee recipiendaire_allocations de type recipiendaire ;
  regle recipiendaire_allocations defini comme
    Complet entite_en_charge_de_l_enfant.
\end{lstlisting}
En cas de résidence alternée de l'enfant au domicile de chacun des parents telle que prévue à l'article 373-2-9 du code civil, mise en oeuvre de manière effective, les parents désignent l'allocataire. Cependant, la charge de l'enfant pour le calcul des allocations familiales est partagée par moitié entre les deux parents soit sur demande conjointe des parents, soit si les parents sont en désaccord sur la désignation de l'allocataire. Un décret en Conseil d'Etat fixe les conditions d'application du présent alinéa.
\begin{lstlisting}
situation EnfantAllocationsFamiliales source loi :
  donnee garde_alternee ;
# on ne formalise pas l'article 373-2-9 pour l'instant
  donnee parent1_garde_alternee de situation Personne ;
  donnee parent2_garde_alternee de situation Personne ;

  fonction est_en_charge parametres
    -- parent de situation Personne
  renvoie booleen:
    selon entite_en_charge_de_l_enfant sous forme
    -- Monoparentale de parent' : parent = parent'
    -- Couple de menage :
      menage.parent1 = parent ou menage.parent2 = parent ;
  assertion condition garde_alternee consequence
    est_en_charge de parent1_garde_alternee ou
    est_en_charge de parent2_garde_alternee ;
  donnee parent_recipiendaire_garde_alternee de situation Personne ;
  regle condition garde_alternee consequence
    recipiendaire_allocations defini comme
      Complet de parent_recipiendaire_garde_alternee ;
  donnee desaccord_designation_allocataire_garde_alternee ;
  donnee demande_conjointe_partage_charge_garde_alternee ;
  donnee recipiendaire_divise_garde_alternee
     de situation RecipiendaireDivise ;
  regle condition garde_alternee et
    (desaccord_designation_allocataire_garde_alternee ou
    demande_conjointe_partage_charge_garde_alternee)
  consequence
  -- recipiendaire_divise_garde_alternee.parent1 defini comme
    parent1_garde_alternee
  -- recipiendaire_divise_garde_alternee.parent2 defini comme
    parent2_garde_alternee
  -- recipiendaire_allocations defini comme
    Divise recipiendaire_divise_garde_alternee ;
  assertion selon recipiendaire_allocations sous forme
  -- Complet de (FamilleMonoparentale de personne) :
    est_en_charge de personne;
  -- Complet de (Couple de couple) :
    Couple de couple = entite_en_charge_enfants
  -- Divise : vrai.
\end{lstlisting}
Lorsque la personne qui assume la charge effective et permanente de l'enfant ne remplit pas les conditions prévues au titre I du présent livre pour l'ouverture du droit aux allocations familiales, ce droit s'ouvre du chef du père ou, à défaut, du chef de la mère.
\begin{lstlisting}
situation EnfantAllocationsFamiliales source loi :
  donnee entite_en_charge_des_enfants_remplit_les_conditions_du_titre_I.
  assertion
    entite_en_charge_des_enfants_remplit_les_conditions_du_titre_I.
# on ne formalise pas pour l'instant, c'est un placeholder.
\end{lstlisting}
Lorsqu'un enfant est confié au service d'aide sociale à l'enfance, les allocations familiales continuent d'être évaluées en tenant compte à la fois des enfants présents au foyer et du ou des enfants confiés au service de l'aide sociale à l'enfance. La part des allocations familiales dues à la famille pour cet enfant est versée à ce service. Toutefois, le juge peut décider, d'office ou sur saisine du président du conseil général, à la suite d'une mesure prise en application des articles 375-3 et 375-5 du code civil ou des articles 15,16,16 bis et 28 de l'ordonnance n° 45-174 du 2 février 1945 relative à l'enfance délinquante, de maintenir le versement des allocations à la famille, lorsque celle-ci participe à la prise en charge morale ou matérielle de l'enfant ou en vue de faciliter le retour de l'enfant dans son foyer.
\begin{lstlisting}
situation EnfantAllocationsFamiliales source loi :
  donnee enfant_confie_au_service_sociaux.
  donnee service_social : Personne.
  regle optionnelle condition enfant_confie_au_service_sociaux
  consequence recipiendaire_allocations defini comme
    Complet de (FamilleMonoparentale de service_social)
\end{lstlisting}

Un décret en Conseil d'Etat fixe les conditions d'application du présent article, notamment dans les cas énumérés ci-dessous :
a) retrait total de l'autorité parentale des parents ou de l'un d'eux ;
b) indignité des parents ou de l'un d'eux ;
c) divorce, séparation de corps ou de fait des parents ;
d) enfants confiés à un service public, à une institution privée, à un particulier.
\begin{lstlisting}
choix couple_ou_partie :
  -- DeuxParents
  -- Parent1
  -- Parent2.

situation EnfantAllocationsFamiliales source loi :
  donnee retrait_autorite_parentale de choix couple_ou_partie ;
  donnee indignite_parents de choix couple_ou_partie ;
  fonction couple_ou_partie_valide parametres
    -- partie de choix couple_ou_partie
  renvoie booleen :
    selon entite_en_charge_de_l_enfant sous forme
    -- FamilleMonoparentale de parent : partie = Parent1
    -- Couple de couple : vrai ;
  asserttion couple_ou_partie_valide retrait_autorite_parentale et
    couple_ou_partie_valide indignite_parents ;
  donnee propriete divorce_parents ;
  donnee enfant_confie_service_public_institution ;
  assertion recipiendaire_allocations fixe par decret ;
  assertion recipiendaire_allocations varie avec
    retrait_autorite_parentale ;
  assertion recipiendaire_allocations varie avec indignite_parents ;
  assertion recipiendaire_allocations varie avec divorce_parents ;
  assertion recipiendaire_allocations varie avec
    enfant_confie_service_public_institution.
\end{lstlisting}

\paragraph{Article L521-3} Chacun des enfants à charge, à l'exception du plus âgé, ouvre droit à partir d'un âge minimum à une majoration des allocations familiales.
\begin{lstlisting}
situation AllocationFamiliales source loi :
  donnee age_minimum_majorations_512_3 de type entier ;
  donnee droits_ouverts_majorations_allocations_familiales ;
  donnee enfant_plus_age de situation EnfantAllocationsFamiliales ;
  regle enfant_plus_age defini comme
    maximum_collection(contexte.enfants, age) ;
  regle condition  existe enfant dans enfants tel que
    enfant.age > age_minimum_majorations_512_3 et
    non (enfant = enfant_plus_age)
  consequence
    droits_ouverts_majorations_allocations_familiales.
\end{lstlisting}

Toutefois, les personnes ayant un nombre déterminé d'enfants à charge bénéficient de ladite majoration pour chaque enfant à charge à partir de l'âge mentionné au premier alinéa.
\begin{lstlisting}
situation AllocationFamiliales source loi :
  donnee nombre_enfants_a_charge_L521_3 de type entier.
  regle condition
    nombre_enfants_a_charge = nombre_enfants_a_charge_L521_3 et
    existe enfant dans enfants tel que
      enfant.age > age_minimum_majorations_512_3
  consequence
    droits_ouverts_majorations_allocations_familiales.
\end{lstlisting}
\end{document}


\clearpage

\section{Questionnaire de satisfaction}



Le langage de programmation présenté ci-dessus est toujours à l'étude, et nous souhaitons que le projet bénéficie le plus tôt possible de retours de professionnels susceptibles d'utiliser ou d'être en contact avec ce nouvel outil. Aussi nous vous proposons de nous faire part de vos retours en remplissant le formulaire page suivante.

Les questions sont posées dans un ordre croissant de difficulté de compréhension du document. N'hésitez pas à faire référence au code dans vos réponses, en citant le numéro de ligne correspondant : chaque ligne de code est en effet annotée d'un numéro sur la gauche.

Une fois le formulaire rempli, enregistrez le PDF sous un autre nom et envoyez-le par email aux auteurs aux adresses suivantes:

\begin{center}
  \href{mailto:denis.merigoux@inria.fr}{\texttt{denis.merigoux@inria.fr}}\\[0.5em]
  \href{mailto:huttner.liane@gmail.com}{\texttt{huttner.liane@gmail.com}}\\[0.5em]
  \href{mailto:nicolas.chataing@ens.fr}{\texttt{nicolas.chataing@ens.fr}}
\end{center}

\label{form}
\begin{Form}
  \begin{center}
    \small
  \textbf{(1) Quel est votre nom et profession ? Vous pouvez également indiquer des détails pertinents sur votre expérience ou parcours.}\\[1em]

    \TextField[multiline=true, width=\textwidth]{}
  \end{center}
  \begin{center}
  \textbf{(2) Est-ce que vous comprenez la démarche ? Si oui, vous paraît-elle justifiée ?}\\[1em]

    \TextField[multiline=true, width=\textwidth]{}
  \end{center}
  \begin{center}
  \textbf{(3) Est-ce que vous arrivez à poser votre regard sur les morceaux de code sans avoir mal à la tête ? Si non, comment améliorer cela ?}\\[1em]

    \TextField[multiline=true, width=\textwidth]{}
  \end{center}
  \begin{center}
  \textbf{(4) Est-ce que vous arrivez à comprendre le sens du code étant donné le guide de lecture au début ? Si non, comment améliorer cela ?}\\[1em]

    \TextField[multiline=true, width=\textwidth, height=12em]{}
  \end{center}
  \begin{center}
  \textbf{(5) Est-ce que vous arrivez à relier le sens de chacun des morceaux de code au sens du texte législatif qu'il est censé annoté ? Si non, comment améliorer cela ?}\\[1em]

    \TextField[multiline=true, width=\textwidth, height=12em]{}
  \end{center}
  \begin{center}
  \textbf{(6) Est-ce que vous pouvez certifier que le code fait bien ce que dit la loi et rien de plus ? Si non, est-ce que vous avez trouvé une erreur dans le code ? }\\[1em]

    \TextField[multiline=true, width=\textwidth, height=24em]{}
  \end{center}
\end{Form}


\end{document}
